\documentclass[a4paper,12pt]{article}
\usepackage{amsmath}
\usepackage{amssymb}
\usepackage{geometry}
\geometry{a4paper, margin=1in}

\begin{document}

\title{Report sull'Algoritmo di Clustering Bilanciato con Nodi Condivisi}
\author{}
\date{}
\maketitle

\section*{1. Introduzione}
Questo algoritmo suddivide \( n \) nodi in \( k \) cluster rispettando i seguenti vincoli:
\begin{itemize}
    \item Ogni cluster deve contenere almeno \( \texttt{min\_shared\_nodes} \) nodi condivisi con altri cluster.
    \item Ogni cluster deve contenere almeno \( \texttt{min\_exclusive\_nodes} \) nodi esclusivi.
    \item Tutti i nodi devono essere assegnati a uno o più cluster.
\end{itemize}

Una caratteristica fondamentale dell'algoritmo è la capacità di gestire situazioni in cui un cluster non soddisfa il numero minimo di nodi condivisi (\( \texttt{min\_shared\_nodes} \)). In tali casi, l'algoritmo forza l'aggiunta di nodi condivisi al cluster interessato considerando i nodi non condivisi degli altri cluster, minimizzando la distanza dal centroide del cluster corrente. Inoltre, durante questo processo, viene garantito che ogni cluster mantenga almeno \( \texttt{min\_exclusive\_nodes} \) nodi esclusivi.

\section*{2. Fasi dell'Algoritmo}

\subsection*{2.1 Selezione dei centroidi iniziali}
\textbf{Scopo:} Selezionare \( k \) centroidi iniziali che siano il più lontani possibile tra loro.\\
\textbf{Algoritmo:}
\begin{enumerate}
    \item Il primo centroide è scelto arbitrariamente.
    \item Iterativamente, si seleziona il nodo più distante dai centroidi già scelti:
    \[
    \texttt{next\_centroid} = \arg\max_i (\texttt{cached\_dists}[i]),
    \]
    dove \texttt{cached\_dists} mantiene le distanze minime dai centroidi già scelti.
\end{enumerate}
\textbf{Complessità:} \( O(k \cdot n) \).

\subsection*{2.2 Popolamento bilanciato dei cluster}
\textbf{Scopo:} Assegnare i nodi ai cluster in modo bilanciato, mantenendo una dimensione uniforme per ciascun cluster.\\
\textbf{Metodo:} L'algoritmo utilizza una strategia basata su heap per popolare i cluster in modo bilanciato. Ecco i passi principali:

\begin{enumerate}
    \item \textbf{Inizializzazione degli heap:}
    \begin{itemize}
        \item Viene costruito un \texttt{heap} (una coda con priorità) per ogni cluster.
        \item Ogni heap contiene i nodi non assegnati, ordinati in base alla distanza dal centroide del cluster corrispondente.
        \item La priorità nell'heap è determinata dalla distanza: il nodo con la distanza minima dal centroide è il primo ad essere estratto.
    \end{itemize}

    \item \textbf{Assegnazione dei nodi ai cluster:}
    \begin{itemize}
        \item Per ogni cluster, viene estratto il nodo con la distanza minima dall'heap del cluster.
        \item Il nodo viene aggiunto al cluster se non è già stato assegnato ad altri cluster.
        \item Una volta assegnato un nodo a un cluster, questo nodo viene rimosso dagli heap degli altri cluster per garantire che non venga assegnato più volte.
    \end{itemize}

    \item \textbf{Controllo delle dimensioni del cluster:}
    \begin{itemize}
        \item Il processo di assegnazione continua iterativamente finché ogni cluster raggiunge la dimensione desiderata (\texttt{cluster\_size}).
        \item Se un heap non ha più nodi da estrarre, il processo passa al cluster successivo.
    \end{itemize}
\end{enumerate}

\textbf{Complessità:} \( O(k \cdot n \log n) \).

\subsection*{2.3 Ricerca nodi condivisi}
\textbf{Scopo:} Garantire che ogni cluster abbia almeno \( \texttt{min\_shared\_nodes} \) nodi condivisi mantenendo il vincolo sui nodi esclusivi.\\
\textbf{Metodo:}
\begin{enumerate}
    \item Si costruisce un \texttt{heap} per ogni cluster, contenente i nodi non condivisi di altri cluster, ordinati in base alla distanza dal centroide del cluster corrente.
    \item Per ogni nodo candidato, si verifica che il cluster di origine abbia un numero sufficiente di nodi esclusivi (\( > \texttt{min\_exclusive\_nodes} \)).
    \item Si seleziona iterativamente il nodo più vicino dal \texttt{heap} e lo si aggiunge come nodo condiviso tra il cluster corrente e il cluster di origine del nodo.
    \item Questo processo continua finché ogni cluster soddisfa il vincolo di \( \texttt{min\_shared\_nodes} \).
\end{enumerate}

\textbf{Modifica introdotta:} Durante l'assegnazione dei nodi condivisi, il controllo aggiuntivo garantisce che i cluster di origine mantengano sempre almeno \( \texttt{min\_exclusive\_nodes} \) nodi esclusivi. Se un cluster ha un numero di nodi esclusivi \( \leq \texttt{min\_exclusive\_nodes} \), i suoi nodi non possono essere selezionati come condivisi.

\textbf{Complessità:} \( O(k \cdot n \log n) \).

\subsection*{2.4 Calcolo dei nodi esclusivi}
\textbf{Scopo:} Identificare i nodi esclusivi di ciascun cluster.\\
\textbf{Metodo:} Per ogni nodo in un cluster, si verifica se non è presente nell'elenco dei nodi condivisi.\\
\textbf{Complessità:} \( O(n) \).

\subsection*{2.5 Visualizzazione avanzata dei cluster}
\textbf{Scopo:} Visualizzare i cluster con un'ulteriore evidenza grafica dei nodi condivisi.\\
\textbf{Metodo:}
\begin{itemize}
    \item Ogni nodo condiviso è rappresentato con due colori, ognuno corrispondente a uno dei due cluster più vicini.
    \item Viene utilizzata la libreria \texttt{matplotlib} e la funzione \texttt{Wedge} per creare un effetto visivo a doppio colore.
\end{itemize}

---

\section*{3. Complessità Complessiva}
La complessità totale dell'algoritmo aggiornato (senza considerare la generazione della matrice delle distanze) è:
\[
O(k \cdot n \log n) \; (\text{popolamento dei cluster}) + O(k \cdot n \log n) \; (\text{forzatura dei nodi condivisi}).
\]
Pertanto, la complessità complessiva è:
\[
O(k \cdot n \log n).
\]

---

\section*{4. Conclusioni}
L'algoritmo di clustering bilanciato aggiornato:
\begin{itemize}
    \item Suddivide i nodi in cluster bilanciati rispettando i vincoli di nodi condivisi ed esclusivi.
    \item Durante la ricerca dei nodi condivisi, utilizza un approccio basato sulla minimizzazione delle distanze, selezionando i nodi più vicini al centroide del cluster corrente tramite heap.
    \item Garantisce che ogni cluster mantenga almeno \( \texttt{min\_exclusive\_nodes} \) nodi esclusivi anche dopo l'assegnazione dei nodi condivisi.
    \item Utilizza strutture dati efficienti (heap) e un processo iterativo ottimizzato per mantenere le prestazioni scalabili.
    \item Fornisce un output visivo avanzato con evidenza grafica dei nodi condivisi, permettendo una migliore interpretazione dei risultati.
\end{itemize}

\end{document}

